\documentclass{article}
\usepackage{mathrsfs}
\usepackage{harpoon}
\usepackage{comment}
%\usepackage{ctex}
\usepackage{color}
\usepackage{amsmath}
\usepackage{mathtools}
\usepackage{simplewick} 
\usepackage{graphicx} % added for subfig
\usepackage{subfigure} % added for subfig
\title{Note for 2pt calculation with Chroma}
\author{Zhipeng Xing and Jinchen He}
\date{}
\usepackage[a4paper,left=20mm,right=20mm,top=20mm,bottom=20mm]{geometry}
\begin{document}
\maketitle
%\tableofcontents
%\pagebreak[4]
\section{Chroma Installation}

\subsection{Download package}

Download necessary packages for Installation from GitHub.

\begin{itemize}
    \item Use "git clone --recursive \dots", "recursive" means after the clone is created, initialize all submodules within, using their default settings.
    \item If the connection to the GitHub is not stable on the server, you are suggested to clone on your local machine, then use "scp" to upload.
\end{itemize}

Package list:

\begin{enumerate}
    \item qmp
    \item qio
    \item qla
    \item qdp
    \item qopqdp
    \item qdpxx
\end{enumerate}

\subsection{Configure and make}

Configure and make in each folder of packages.

\begin{itemize}
    \item The whole process can be divided into 7 parts (6 packages above and chroma), so that you can locate the errors conveniently.
    \item "export PATH=\dots:\$PATH", makes environment variables available to other programs called from bash. 
    \item "autoreconf -vi": used to update generated configuration files, "-v" means verbosely reporting processing, "-i" means copying missing auxiliary files.
    \item "./configure", you can use "./configure –help" to see the options
    \item "./autogen.sh"
\end{itemize}

\section{Source code}

\subsection{Plug in packages}

Users are allowed to write some plug in packages and register in the Chroma, so that those packages can be used.

\subsection{Make}

\begin{itemize}
    \item Makefile
    \item make.sh 
\end{itemize}

\section{2pt calculation}

\subsection{Perl script}

Used to print the .xml file as the input for Chroma.

Write perl script as the structure in "xxx.h".

\subsection{Inline 2pt.cc}




\subsection{Add new plug in packages}

If you want to use a new plug in package in the Chroma for calculation, you should:

\begin{enumerate}
    \item Write the .cc file and .h file.
    \item Put two files above into the source code folder.
    \item In the source code folder, add '\#include "inline\_xxx.h" ' and "foo \&= InlinexxxEnv::registerAll();" into "chroma.cc".
    \item In the source code folder, add "inline\_2pt.h" and "inline\_2pt.o" into "Makefile".
    \item "bash make.sh" again
    \item Update your .pl file to use the new plug in package, and remake the soft link of "chroma" in the same path as .pl file.
    \item "sbatch xxx.sh" again.
\end{enumerate}


\end{document}